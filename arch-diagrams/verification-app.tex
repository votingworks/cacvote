\documentclass{article}
\input{macros}

\begin{document}
Suppose we have a Bulletin Board with something like the following operations:
    \begin{description}
        \item[post(m, id)] post item $m$, authenticated with ID. Suppose that this returns some kind of \emph{receipt}, e.g. an ack signed by the server, to promise to upload this to the BB. (Note this is \emph{not} an inclusion proof - it's a promise to post it on the BB.) This receipt requires some pre-loaded validation data, e.g. the server or BB's sig verification key.
        \item[get\_inclusion\_proof(m)] get a proof that $m$ is included.
        \item[get\_all] get all the contents.
    \end{description}
    The exact details can vary. For example, Andrew Conway's bulletin board (**add details).

    Suppose that at the close of polls (or shortly afterwards given some grace period) the BB publishes a root hash that includes everything that was uploaded before the deadline. Call this $H1$. We need to assume some out-of-band way for everyone to agree on $H1$ without trusting the BB or server.



    This verification app can be used in 4 modes:
    \begin{itemize}
        \item Voter mode: the voter scans their own sticker and continues to poll the bulletin board for its inclusion;
        \item \remoteVotingCenter{} authority mode: an authority at the \remoteVotingCenter{} scans everyone's stickers, and continues to poll the BB for the inclusion of \emph{all} of them;
        \item \localVotingCenter{} authority mode: an authority at the \localVotingCenter{}, responsible for receiving the paper votes, scans the incoming stickers and verifies the signatures ...;
        \item \localVotingCenter{} public/auditor mode: an auditor at the \localVotingCenter{} scans the stickers and verifies the signatures in the context of an audit. This usage may also poll the BB and verify other aspects.
    \end{itemize}

    We also have two protocol options, which are relevant to the verification options:
    \BBInclusionCheck{} and \StickerBBUpload{}. These alter the details of how the verification app checks for the presence of a vote on the BB.

    This document details the implementation steps for the Verification summary described in Section 4.5 of the main Merge paper. We emphasise the implementation details---procedural responses are detailed elsewhere.

\subsection{Subroutines}
Each mode uses a subset of the following subroutines:

\begin{enumerate}
    \item \textbf{Sticker signature verification}:
\begin{itemize}
    \item Input: scanned QR code,  CA certificate, election Id.
    \item Output: CAC ID and Ballot Hash if 
    \begin{itemize}
        \item the scanned Election ID is as expected, 
        \item the certificate chain is valid, and
        \item the signature of the message (CAC ID, Election ID,Ballot Hash) is verified,
    \end{itemize}
    and ``invalid signature'' otherwise.
\end{itemize}
\label{Sticker signature verification}
    \item \textbf{Vote \BBInclusionCheck{}}:

\begin{itemize}
    \item    Input: Voter CAC ID, Election Id, Ballot hash, $H1$
    \item     Do: Query the BB  searching for a vote corresponding to the provided CAC ID and Election ID,
    \item Output: 
    \begin{itemize}
        \item ``NULL'' if the query 's output is NULL,
        \item ``True'' if (1) the query 's output is not NULL, (2) the inclusion proof is successfully verified, and
        (3) The CAC ID, Election ID and the ballot hash computed from the query's output, are as expected, or
        \item ``False'' otherwise.
    \end{itemize}

\end{itemize}
    Note that this verifies inclusion---it does not check that the included vote is valid.

\label{Vote BB presence-BB Inc}
    \item \textbf{\StickerBBUpload{}}:
\begin{itemize}
    \item Setup: receipt validation info,
    \item Input: Sticker data with valid signature
    \item Do: Post the sticker data to the BB; validate receipt.
\end{itemize}
\label{Vote BB presence-upload}

    \item \textbf{Sticker \BBInclusionCheck{}}
    \begin{itemize}
        \item Input: includes $H1$.
        \item TODO - similar to Vote BB inclusion check.
    \end{itemize}

    \item \textbf{Valid Vote Arrival}:
\begin{itemize}
    \item Input: List of enrolled CAC IDs, Election ID, CAC ID and Election ID of the arrived ballot
    \item Output: 
    \begin{itemize}
        \item ``Valid'' if     (1) The CAC ID is on the enrolled voter list, (2) Election ID of the arrived ballot is as expected, and (3) The CAC ID is not already marked as arrived on the BB, or
        \item ``Invalid'' otherwise.

    \end{itemize}
\end{itemize}
\label{Valid Vote Arrival}
    \item \textbf{Arrival BB Notification}:
\begin{itemize}
    \item Input: CAC ID, Election ID.
    \item Do: Mark the ballot with matching CAC ID and Election ID on the BB.

\end{itemize}
\label{BB Notification of Valid Vote Arrival}
\item \textbf{Decryption proof processing}:
\begin{itemize}
    \item Input: An encrypted ballot  (or an encrypted tally), a cleartext ballot (or a cleartext tally), proof of decryption, aggregated public key, cryptography public parameters
    \item Output: ``True'' if proof of decryption is successfully verified or ``False'' otherwise.
\end{itemize}
More information can be found in ``verification 9'' and ``verification 10'' in pages 40 and 41 of the ''ElectionGuard, Design Specification,
Version 2.0.0''.
\label{Decryption proof processing}
\end{enumerate}

    

\subsection{App modes}
\subsubsection{Voter mode}

\textbf{\BBInclusionCheck{} mode}:


App configuration data: CA certificate, Election Id, query interval, query number


Routine:
\begin{enumerate}
    \item Scan the sticker
    \item Run the subroutine \ref{Sticker signature verification} (\StickersignatureVerification{}) with scanned QR code, CA certificate and election ID as input. If the Subroutine \ref{Sticker signature verification}'s output $\neq\;$ ``invalid signature'', continue. Otherwise, halt.
    \item Continuously run the subroutine \ref{Vote BB presence-BB Inc} (\BBInclusionCheck{}) at intervals specified by the query interval, with scanned CAC ID, Election ID and Ballot Hash as input until either (1) Subroutine \ref{Vote BB presence-BB Inc}'s outputs ``True'' or ``False'' or (2) the number of complete queries is already equal to the specified query number. 
    \item Output 
    \begin{itemize}
        \item ``Included'' if Subroutine \ref{Vote BB presence-BB Inc}'s output is ``True'',
        \item ``Not included'' if Subroutine \ref{Vote BB presence-BB Inc}'s output is ``False'', or
        \item ``Time out'' otherwise.
    \end{itemize}
    
\end{enumerate}

\textbf{\StickerBBUpload{}{} mode}

App configuration data: CA certificate, Election Id


Routine:
\begin{enumerate}
    \item Scan the sticker
    \item Run the subroutine \ref{Sticker signature verification} (\StickersignatureVerification{}) with scanned QR code, CA certificate and election ID as input. If the Subroutine \ref{Sticker signature verification}'s output $\neq\;$ ``invalid signature'', continue. Otherwise, halt.
    \item Run the subroutine \ref{Vote BB presence-upload} (\StickerBBUpload{}).
\end{enumerate}

\subsubsection{\remoteVotingCenter{} authority mode}


App configuration data: CA certificate, Election Id, query interval, query number

\textbf{\BBInclusionCheck{} mode}:


Routine:
\begin{enumerate}
    \item Scan all stickers
    \item For each scanned sticker:
    \begin{itemize}
        \item Run the subroutine \ref{Sticker signature verification} (\StickersignatureVerification{}) with scanned QR code, CA certificate and election ID as input. If the Subroutine \ref{Sticker signature verification}'s output $\neq\;$ ``invalid signature'', continue. Otherwise, halt.
    \item Continuously run the subroutine \ref{Vote BB presence-BB Inc} (\BBInclusionCheck{}) at intervals specified by the query interval, with scanned CAC ID, Election ID and Ballot Hash as input until either (1) Subroutine \ref{Vote BB presence-BB Inc}'s outputs ``True'' or ``False'' or (2) the number of complete queries is already equal to the specified query number. 
    \item Output 
    \begin{itemize}
        \item ``Included'' if Subroutine \ref{Vote BB presence-BB Inc}'s output is ``True'',
        \item ``Not included'' if Subroutine \ref{Vote BB presence-BB Inc}'s output is ``False'', or
        \item ``Time out'' otherwise.
    \end{itemize}
    \end{itemize}
\end{enumerate}

\textbf{\StickerBBUpload{}{} mode}

Routine:
\begin{enumerate}
    \item Scan all the sticker:
    \item For each scanned sticker:
    \begin{itemize}

        \item Run the subroutine \ref{Sticker signature verification} (\StickersignatureVerification{}) with scanned QR code, CA certificate and election ID as input. If the Subroutine \ref{Sticker signature verification}'s output $\neq\;$ ``invalid signature'', continue. Otherwise, halt.
        \item Run the subroutine \ref{Vote BB presence-upload} (\StickerBBUpload{}).
    \end{itemize}
\end{enumerate}

\subsubsection{\localVotingCenter{} authority mode}
App configuration data: CA certificate, Election Id

Routine:
\begin{enumerate}
    \item Scan the sticker
    \item Run the subroutine \ref{Sticker signature verification} (\StickersignatureVerification{}) with scanned QR code, CA certificate and election ID as input. If the Subroutine \ref{Sticker signature verification}'s output $\neq\;$ ``invalid signature'', continue. Otherwise, halt.
    \item Run the subroutine \ref{Vote BB presence-BB Inc} (\BBInclusionCheck{}). If subroutine \ref{Vote BB presence-BB Inc}'s output equals ``True'', continue. Otherwise halt.
    \item Run the subroutine \ref{Valid Vote Arrival} (\ValidVoteArrival{}). If subroutine \ref{Valid Vote Arrival}' output equals ``Valid'', continue. Otherwise halt.
    \item Run the subroutine \ref{BB Notification of Valid Vote Arrival} (\ArrivalBBNotification{}).
\end{enumerate}
\subsubsection{\localVotingCenter{}  public/auditor mode:}
Application in this mode should have multiple functionalities as follows.


\textbf{Sticker processing}

App configuration data: CA certificate, Election Id

Routine:
\begin{enumerate}
    \item Scan the sticker
    \item Run the subroutine \ref{Sticker signature verification} (\StickersignatureVerification{}) with scanned QR code, CA certificate and election ID as input.
\end{enumerate}

\textbf{Decryption processing}

Routine: Run the subroutine \ref{Decryption proof processing} (\DecryptionProof{}).

\subsection{Universal verification}
    This is a lot of work. Prob not great for a mobile app, but maybe OK as an option...




\end{document}