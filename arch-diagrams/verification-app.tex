\documentclass{article}
\input{macros}

\begin{document}
Suppose we have a Bulletin Board with something like the following operations:
    \begin{description}
        \item[post(m, id)] post item $m$, authenticated with ID. Suppose that this returns some kind of \emph{receipt}, e.g. an ack signed by the server, to promise to upload this to the BB. (Note this is \emph{not} an inclusion proof - it's a promise to post it on the BB.) This receipt requires some pre-loaded validation data, e.g. the server or BB's sig verification key.
        \item[get\_inclusion\_proof(m)] get a proof that $m$ is included.
        \item[get\_all] get all the contents.
    \end{description}
    The exact details can vary. For example, Andrew Conway's bulletin board (\url{https://github.com/RightToAskOrg/bulletin-board})
    has a notion of explicitly asking for a new root hash, relative to which the inclusion proof is produced. Then each
    inclusion proof is a compound of the proof that the item is included in its root hash, and a proof that that root hash
    is included in the current root hash.

    Suppose that at the close of polls (or shortly afterwards given some grace period) the BB publishes a root hash that includes everything that was uploaded before the deadline. Call this $H1$. We need to assume some out-of-band way for everyone to agree on $H1$ without trusting the BB or server.

    This verification app can be used in 4 modes:
    \begin{itemize}
        \item Voter mode: the voter scans their own sticker and continues to poll the bulletin board for its inclusion;
        \item \remoteVotingCenter{} authority mode: an authority at the \remoteVotingCenter{} scans everyone's stickers, and continues to poll the BB for the inclusion of \emph{all} of them;
        \item \localVotingCenter{} authority mode: an authority at the \localVotingCenter{}, responsible for receiving the paper votes, scans the incoming stickers, verifies the signatures and verifies their inclusion on the BB.
        \item \localVotingCenter{} public/auditor mode: an auditor at the \localVotingCenter{} scans the stickers and verifies the signatures in the context of an audit. This usage may also poll the BB and verify other aspects.
    \end{itemize}

    We also have two protocol options, which are relevant to the verification options:
    \BBInclusionCheck{} and \StickerBBUpload{}. These alter the details of how the verification app checks for the presence of a vote on the BB.

    This document details the implementation steps for the Verification summary described in Section 4.5 of the main Merge paper. We emphasise the implementation details---procedural responses are detailed elsewhere.

\section{Subroutines}
Each mode uses a subset of the following subroutines:

\begin{enumerate}
    \item \textbf{Sticker signature verification}:
\begin{itemize}
\item Setup: Signature validation info. (e.g., CA certificate), election ID
    \item Input: scanned QR code
    \item Output: ``Valid Signature'' if 
    \begin{itemize}
        \item the scanned election ID is as expected, 
        \item the certificate chain is valid, and
        \item the signature of the message (CAC ID, election ID,ballot hash) is verified,
    \end{itemize}
    and ``Invalid Signature'' otherwise.
\end{itemize}
\label{Sticker signature verification}
    \item \textbf{Vote \BBInclusionCheck{}}:

\begin{itemize}
    \item    Input: Voter CAC ID, election ID, ballot hash, $H1$
    \item     Do: Query the BB  searching for a vote corresponding to the provided CAC ID and election ID,
    \item Output: 
    \begin{itemize}
       % \item ``NULL'' if the query 's output is NULL,
        \item ``True'' if (1) the query 's output is not NULL, (2) the inclusion proof is successfully verified against the root hash $H1$, and
        (3) The CAC ID, election ID and the ballot hash computed from the query's output, are as expected, or
        \item ``False'' otherwise.
    \end{itemize}

\end{itemize}
    Note that this verifies inclusion---it does not check that the included vote is valid.

\label{Vote BB presence-BB Inc}
    \item \textbf{\StickerBBUpload{}}:
\begin{itemize}
    \item Setup: receipt validation info (e.g., CA certificate)
    \item Input: Sticker data with valid signature
    \item Do: Post the sticker data to the BB and validate its receipt (i.e., validate BB's signature on CAC ID, election ID and ballot hash).
    \item Output: 
    \begin{itemize}
        \item ``Valid Recipt'': If the receipt is successfully validated,
        \item ``Upload Failed'', otherwise.
    \end{itemize}
\end{itemize}
\label{Vote BB presence-upload}

    \item \textbf{Sticker \BBInclusionCheck{}}
    \begin{itemize}
        \item Input: receipt, $H1$.
        \item Do: Query the BB  to find data that matches the information on the receipt.
        \item Output: 
        \begin{itemize}
            \item ``True'' if (1) the query 's output is not NULL, (2) the inclusion proof is successfully verified against the root hash $H1$ and the information on the receipt, or
            \item ``False'' otherwise.
        \end{itemize}
    \end{itemize}
\label{sticker inclusion check}
    \item \textbf{Valid Vote Arrival}:
\begin{itemize}
    \item Input: List of enrolled CAC IDs, election ID, CAC ID and election ID of the arrived ballot
    \item Output: 
    \begin{itemize}
        \item ``Valid'' if     (1) The CAC ID is on the enrolled voter list, (2) The CAC ID's vote is not already marked as arrived on the BB, or
        \item ``Invalid'' otherwise.

    \end{itemize}
\end{itemize}
\label{Valid Vote Arrival}
    \item \textbf{Arrival BB Notification}:
\begin{itemize}
    \item Input: CAC ID, election ID.
    \item Do: Mark the ballot with matching CAC ID and election ID on the BB.

\end{itemize}
\label{BB Notification of Valid Vote Arrival}
\item \textbf{Decryption proof processing}:
\begin{itemize}
    \item Input: An encrypted ballot  (or an encrypted tally), a cleartext ballot (or a cleartext tally), proof of decryption, aggregated public key, cryptography public parameters
    \item Output: ``True'' if proof of decryption is successfully verified or ``False'' otherwise.
\end{itemize}
More information can be found in ``verification 9'' and ``verification 10'' in pages 40 and 41 of the ''ElectionGuard, Design Specification,
Version 2.0.0''.
\label{Decryption proof processing}
\end{enumerate}

    

\subsection{App modes}
\subsubsection{Voter mode}

\textbf{\BBInclusionCheck{} mode}:


App configuration data: Signature validation info (i.e., CA certificate), election ID, root hash release time 


Routine:
\begin{enumerate}
    \item Scan the sticker
    \item Run subroutine~\ref{Sticker signature verification} (\StickersignatureVerification{}) with scanned QR code as input. If Subroutine \ref{Sticker signature verification}'s output $=$ ``Valid Signature'', continue. Otherwise, halt.
    \item Wait until the root hash release time, then obtain the root hash $H1$ once it is released.
    \item Run subroutine~\ref{Vote BB presence-BB Inc} (Vote \BBInclusionCheck{}) with  $H1$ and the scanned CAC ID, election ID and ballot hash as input.
    %\item Continuously run the subroutine \ref{Vote BB presence-BB Inc} (\BBInclusionCheck{}) at intervals specified by the query interval, with scanned CAC ID, Election ID and Ballot Hash as input until either (1) Subroutine \ref{Vote BB presence-BB Inc}'s outputs ``True'' or ``False'' or (2) the number of complete queries is already equal to the specified query number. 
    %\item Output 
    %\begin{itemize}
        %\item ``Included'' if Subroutine \ref{Vote BB presence-BB Inc}'s output is ``True'',
        %\item ``Not included'' if Subroutine \ref{Vote BB presence-BB Inc}'s output is ``False'', or
        %\item ``Time out'' otherwise.
    %\end{itemize}
    
\end{enumerate}

\textbf{\StickerBBUpload{}{} mode}

App configuration data: Receipt validation info (i.e., CA certificate), election ID, root hash release time


Routine:
\begin{enumerate}
    \item Scan the sticker
    \item Run subroutine~\ref{Sticker signature verification} (\StickersignatureVerification{}) with scanned QR code, CA certificate and election ID as input. If Subroutine~\ref{Sticker signature verification}'s output $=$ ``Valid signature'', continue. Otherwise, halt.
    \item Run subroutine~\ref{Vote BB presence-upload} (\StickerBBUpload{}) to obtain a valid receipt.
    \item Wait until the root hash release time, then obtain the root hash $H1$ once it is released.
    \item Run subroutine~\ref{sticker inclusion check} (Sticker \BBInclusionCheck{}) with  $H1$ and the receipt as the input.
\end{enumerate}

\subsubsection{\remoteVotingCenter{} authority mode}


App configuration data: CA certificate, election ID, root hash release time

\textbf{\BBInclusionCheck{} mode}:


Routine:
\begin{enumerate}
    \item For each sticker:
    \begin{enumerate}
        \item Scan the sticker
        \item Run subroutine~\ref{Sticker signature verification} (\StickersignatureVerification{}) with scanned QR code, CA certificate and election ID as input.
    \end{enumerate}
    \item Wait until the root hash release time, then obtain the root hash $H1$ once it is released.
    \item For each scanned sticker with ``valid Signature'' output from step 1-b, run subroutine~\ref{Vote BB presence-BB Inc} (Vote \BBInclusionCheck{}) with  $H1$ and the scanned CAC ID, election ID and ballot hash as input.
    \item Output:
    \begin{itemize}
        \item ``Passed'',  if Subroutine~\ref{Sticker signature verification}'s and~\ref{Vote BB presence-BB Inc}'s outputs for all stickers are ``Valid Signature'' and ``True'', respectively.
        \item ``Rejected'', otherwise.
    \end{itemize}
\end{enumerate}


\textbf{\StickerBBUpload{}{} mode}

Routine:
\begin{enumerate}
    \item For each sticker:
    \begin{enumerate}
        \item Scan the sticker
        \item Run Subroutine~\ref{Sticker signature verification} (\StickersignatureVerification{}) with scanned QR code, CA certificate and election ID as input.
        \item If Subroutine~\ref{Sticker signature verification}'s output is ``Valid Signature'', run Subroutine~\ref{Vote BB presence-upload} (\StickerBBUpload{}).
    \end{enumerate}
    \item Wait until the root hash release time, then obtain the root hash $H1$ once it is released.
    \item For each uploaded sticker with ``Valid Receipt'' output from step 1-c, run subroutine~\ref{sticker inclusion check} (Sticker \BBInclusionCheck{}) with  $H1$ and the receipt received in step 1-c as input.
    \item Output
    \begin{itemize}
        \item ``Passed'',  if Subroutine~\ref{Sticker signature verification}'s, \ref{Vote BB presence-upload}'s and~\ref{sticker inclusion check}'s outputs for all stickers are ``Valid Signature'' and ``Valid Receipt'' and ``True'' respectively.
        \item ``Rejected'', otherwise.
    \end{itemize}
\end{enumerate}




\subsubsection{\localVotingCenter{} authority mode}
App configuration data: CA certificate, Election Id

Routine:
\begin{enumerate}
    \item Scan the sticker
    \item Run subroutine \ref{Sticker signature verification} (\StickersignatureVerification{}) with scanned QR code as input. If the Subroutine \ref{Sticker signature verification}'s output $=$ ``Valid Signature'', continue. Otherwise, halt.
    \item Run subroutine \ref{Vote BB presence-BB Inc} (Vote \BBInclusionCheck{}) or Subroutine \ref{sticker inclusion check} (Sticker \BBInclusionCheck{}). If its output equals ``True'', continue. Otherwise halt.
    \item Run subroutine \ref{Valid Vote Arrival} (\ValidVoteArrival{}). If subroutine \ref{Valid Vote Arrival}' output equals ``Valid'', continue. Otherwise halt.
    \item Run subroutine \ref{BB Notification of Valid Vote Arrival} (\ArrivalBBNotification{}).
\end{enumerate}
\subsubsection{\localVotingCenter{}  public/auditor mode:}
Application in this mode should have multiple functionalities as follows.


\textbf{Sticker processing}

App configuration data: CA certificate, Election Id

Routine:
\begin{enumerate}
    \item Scan the sticker
    \item Run subroutine \ref{Sticker signature verification} (\StickersignatureVerification{}) with scanned QR code as input.
    \item Run subroutine \ref{Vote BB presence-BB Inc} (Vote \BBInclusionCheck{}) or Subroutine \ref{sticker inclusion check} (Sticker \BBInclusionCheck{}).

\end{enumerate}

\textbf{Decryption processing}

Routine: Run subroutine \ref{Decryption proof processing} (\DecryptionProof{}).

\subsection{Universal verification}
    This is a lot of work. Prob not great for a mobile app, but maybe OK as an option...

If \StickerBBUpload{} is used, Universal verification must include verifying the presence of a corresponding ballot for
every uploaded sticker.

    \subsection{Implementation ideas}
    Doing the final BB check requires knowing when the election should close, and initiating a download. On Android, it looks as if the AlarmManager class would help: \url{https://developer.android.com/develop/background-work/services/alarms/schedule}.


\end{document}